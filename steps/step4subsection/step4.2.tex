\subsection{Coupling between components}
            According to the definition
            \footnote{Coupling definition: \cb{
            \url{https://en.wikipedia.org/wiki/Coupling\_(computer\_programming)}}}:
            \begin{adjustwidth}{1.5cm}{}
                \textit{coupling is the degree of interdependence between software modules; a measure of how closely connected two routines or modules are; the strength of the relationships between modules.}
            \end{adjustwidth}
            Tight coupling in a software system can be associated with more interdependency between components, more coordination and more information flow; on the other hand, low coupling can be associated with less interdependency, coordination and information flow. \\
            Tightly coupled systems can have the following characteristics, often seen as disadvantages:
            \begin{enumerate}
                \item A change in one module can cause a ripple effect of changes in other modules 
                \item Assembly of modules will take considerably longer time thanks to the increased interdependency between them and the high flow of information
                \item Using a single module as a standalone might be difficult due to other modules also being needed.
            \end{enumerate}
            Looking at these characteristics, we can motivate the general idea that good design usually is related to loose coupling in the system.\\\\
            In order to analyze the design quality of the system, we have to analyze the coupling between its components. As such, we will make use of two metrics, namely \texttt{FANIN} and \texttt{FANOUT}. These two metrics were borrowed from logic circuit design, and they represent the number of occurrences of a class (how many times a class is being called) and the number of classes a certain class calls in its definition. They can be used to indicate the extent to which a component is independent, as well as how much responsibility is has.\\
            When describing packages, we can talk about two types of coupling:
            \begin{enumerate}
                \item \textbf{Afferent coupling AC}\\
                The number of packages that depend upon classes in a certain package. It indicates the package's responsibility.
                \item \textbf{Efferent coupling EC}\\
                The number of packages that the classes in a package depend upon. It indicates the package's independence.
            \end{enumerate}